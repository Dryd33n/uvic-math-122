\documentclass{article}
\usepackage{graphicx} % Required for inserting images
\usepackage[legalpaper, portrair, margin=1in]{geometry}
\usepackage{amssymb}

\title{Math 122 I.C.A 6}
\author{Dryden Bryson}
\date{October 2024}

\begin{document}

\maketitle

\newpage
\section*{Question 1:}
Let $S$ be a arbitrary subset: $S\in \mathcal{P}(A\cap B)$, thus we can say that every element of the arbitrary set $S$ is a element of the intersection of $A$ and $B$, thus $S \subseteq A \cap B$.\\

From the above we can say that every element of our arbitrary set $S$ must be an element of the set $A$ \textbf{and} $B$, by the definition of the intersection, thus: $S\subseteq A$ \textbf{and} $S\subseteq B$\\

Since $S \subseteq A$ and $S \subseteq B$ by the definition of the powerset: $$S\in\mathcal{P}(A) \textbf{   and   }  S\in\mathcal{P}(B)$$


\newpage
\section*{Question 2:}
Let the set $A=\{1\}$ and $B=\{2\}$, thus: 


\begin{table}[htp]
    \centering
    \begin{tabular}{c}
        $P= A\oplus B = \{1,2\}$ \\\\ $Q=A\cup B = \{1,2\}$
    \end{tabular}
\end{table}

By the definition of a proper subset, all the elements of $P$ must be in $Q$ and $Q$ must contain an element which is not in $P$. This means that $A \subset B$ but $A$ is not a proper subset of $B$. \\\\We can conclude that it is not always true that $A \oplus B \subsetneqq A \cup B$, or more precisely, it is not true when $A \cap B = \emptyset$.

\end{document}
