\documentclass{article}
\usepackage{graphicx} % Required for inserting images
\usepackage{amsmath}
\usepackage{amssymb}

\title{MATH 122 I.C.A 4}
\author{Dryden Bryson}
\date{September 2024}

\begin{document}

\maketitle
\newpage
\section*{I.C.A 4}
We have:

\begin{table}[htp]
\centering
    \begin{tabular}{cc}
       1. & $\neg r\leftrightarrow \neg q$\\ 
       2. & $\neg q\to p$ \\ 
       3. & $\neg p$ \\ 
        \hline
       4. & $\therefore r\land \neg p$
    \end{tabular}
\end{table}
 \\
To prove the validity of the argument:

\begin{table}[htp]
    \centering
    \begin{tabular}{ccl}
       5.  & $(\lnot r\rightarrow \lnot q) \land (\lnot q\rightarrow \lnot r)$ & Definition of Bi-conditional (1)\\
       6.  & $\lnot r \rightarrow \lnot q$ & Conjunctive Simplification (5)\\
       7.  & $q\rightarrow r$ & Contrapositive (6)\\
       8.  & $\lnot p \rightarrow q$ & Contrapositive (2)\\
       9.  & $q$ & Modus Ponens (3 \& 8)\\
       10. & $r$ & Modus Ponens (9 \& 7)\\
       11. & $r \land \lnot p$ & Conjunction introduction (10 \& 3)\\
    \end{tabular}
\end{table}
 \\
Using inference rules and logical equivalences we show on line 11 that the argument is valid by inferring $r \land \lnot p$ from the given premises. 


\end{document}
