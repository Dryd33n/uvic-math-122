\documentclass{article}
\usepackage[legalpaper, portrait, margin=0.5in]{geometry}
\usepackage{amsmath}
\usepackage{amssymb}

\title{I.C.A 11}
\author{Dryden Bryson}
\date{November 8th 2024}

\begin{document}

\maketitle
\section*{Question 1:}
In order to find the GCD with the euclidian algorithm we need to perform repeated division and taking the remainder, thus we perform:
$$\text{gcd}(731,578)$$
$$\begin{aligned}
    731&=578\times 1+153\\
    578&=153\times 3+119\\
    153&=119\times 1+34\\
    119&=34\times 3+17\\
    34&=17\times 2+0\\
\end{aligned}$$
Since we have that: $34 \div 17$ has a remainder of zero we conclude that: $$\text{gcd}(731,578)=17$$
\section*{Question 2:}
We need to find integers $x$ and $y$ such that: $$17=731x+578y$$
We will use our work from the euclidian algorithm starting with the last division statement with a non-zero remainder. We have that: $$119=34\times 3+17 \;\;\;\;\rightarrow\;\;\;\; 17= 119-3\times 34$$
Substituting in from equation 3 to represent 17 in terms of 119 and 153 we have that: 
$$\begin{aligned}
    17&= 119-3\times 34\\
    &=119-3(153-119)\\
    &=119-3\times 153+3\times 119 \\
    &=4\times 119-3\times 153
\end{aligned}$$
Now we substitute from equation 2 to represent 119 in terms of 578 and 153
$$\begin{aligned}
    17 &=4\times 119-3\times 153\\
    &=4(578-3\times 153)-3\times 153\\
    &= 4\times 578-12\times 153-3\times 153\\
    &=4\times 578 -15\times 153
\end{aligned}$$
Now we substitute from the first question to represent 153 in terms of 731 and 578, we have that:
$$\begin{aligned}
    17 &=4\times 578 -15\times 153\\
    &= 4\times 578-15(731-578)\\
    &= 4\times 578-15\times 731+15\times 578\\
    &= 19\times 578-15\times 731
\end{aligned}$$
Now that 17 is represented in terms of 578 and 731 we have that: $$\begin{aligned}
    17=731x+578y\\
    x=-15\;\;\;\;\;\;\;y=19
\end{aligned}$$
\section*{Question 3:}
We use the relation: $$\text{lcm}(a,b)=\frac{|a\cdot b|}{\text{gcd}(a,b)}$$
Thus we have that: $$\text{lcm}(731,578)=\frac{|731\cdot 578|}{17}=24854$$

\end{document}